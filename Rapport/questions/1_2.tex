Nous allons maintenant nous attarder sur l'aspect théorique de l'algorithme de 
Metropolis-Hastings avant de passer à l'aspect pratique.

\subsubsection{}
Soit $\pi_0$ une distribution initiale d'une chaîne de Markov invariante dans le temps et $Q$ la
matrice de transition :
\begin{equation*}
  \forall i,j \in \{1,...,N\} : \pi_0(i)[Q]_{i,j} = \pi_0(j)[Q]_{j,i}
\end{equation*}

Si $\pi_0$ est une distribution stationnaire de notre chaîne de Markov alors, avec $Q$ la matrice de
transition :
\begin{equation}
  [\pi_0 Q]_j = [\pi_0]_j
\end{equation}

Partons du membre de gauche :

\begin{align*}
  [\pi_0 Q]_j &= \sum_{k = 1}^N \pi_k Q_{k,j} \hspace{2cm} \text{(définition du produit scalaire)}\\
              &= \sum_{k = 1}^N \pi_j Q_{j,k} \hspace{2cm} \text{(par hypothèse des équations de balance détaillées)}\\
              &= [\pi_0]_j \underbrace{\sum_{k=1}^N Q_{j,k}}_\textrm{=1} \hspace{2cm} \text{(définition de la matrice de transition)}\\
              &= [\pi_0]_j \hspace{2cm} (\pi \text{est donc bien une distribution stationnaire})
\end{align*}

On peut finalement conclure sur l'unicité de $\pi_0$. Pour que la distribution stationnaire soit unique, il faut que le chaîne de Markov soit irréductible, c'est-à-dire que 
si on représente notre chaîne selon un graphe avec un poids par arête représentant les probabilités, celui-ci doit être connexe.

\subsubsection{}
